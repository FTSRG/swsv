\chapter{Verifying specifications and requirements}
\label{cha:specifications}

\section*{Learning outcomes}

\begin{enumerate}
    \item Explain the properties and good practices of textual requirements (K2)
    \item Recall the different types of review processes (K1)
    \item List typical review criteria for requirements and specifications (K1)
    \item Perform review of requirements and specifications (K3)
    \item Perform checking of UML state machines for completeness and unambiguousness (K3)
\end{enumerate}

\section{Introduction}

Motivation for verifying specifications and reviews


\section{Requirements}

Definition of requirement:

\begin{quote}
    ``A condition or capability needed by a user to solve a problem or achieve an objective'' \cite{ieee-24765}
\end{quote}

\begin{quote}
   ``A condition or capability that must be met or possessed by a system, system component, product, or service to satisfy an agreement, standard, specification, or other formally imposed documents'' \cite{ieee-24765}
    \end{quote}


\section{Review process and criteria}


\section{Verifying state machines}

UML state machine syntax and semantics \cite{omg-uml2} (For interested readers see \cite{pinter-phd})

\section{Hungarian terms}

\begin{table}[ht]
    \centering
    \small
    \caption{Hungarian terms for verifying specifications chapter}
    \begin{tabular}{ll}
        \toprule
        \textbf{English} & \textbf{Hungarian} \\
        \midrule
        impact analysis & hatáselemzés \\
        inspection & inspekció \\
        peer review & egyenrangú felülvizsgálat \\
        requirement & követelmény \\
        review & felülvizsgálat \\
        scribe & jegyzőkönyv vezető \\
        specification & specifikáció \\
        stakeholder & érdekelt \\
        traceability & nyomonkövethetőség \\
        use case & használati eset \\
        walkthrough & átvizsgálás \\
        \bottomrule
        \end{tabular}
        \label{tab:overview:hungarian-terms-specifications}
        \end{table} 